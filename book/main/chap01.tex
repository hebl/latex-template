\part{基础知识}

第一部分是基础知识

\chapter{序论}
\thispagestyle{empty}


\section{绪论}

曾经击杀过王侯的玄龟,其宝术自然惊人,足以算是镇教神通,居然能在拍卖会上出现,让石昊心中火热,更加期待。

“避火珠一枚,虽然有缺,但也算是近年来难得的珍宝。”在那台上,白发老者展现一枚火红的珠子。

“这可是奇珍,真正的瑰宝,大家仔细观看,无论是用来炼宝具,还是用来收藏,都价值惊人。”那个曲线起伏的女子补充道。

避火珠,自然算是天地间的宝珠,这一颗能有鸽卵那么大,通体赤红,犹如火钻,晶莹透亮。遗憾的是,上面有几道裂痕,更有一处缺失了一块,不够完美无瑕。

这种东西看起来只能避火,缺少更多的妙用,似乎功能单一。然而,其真正价值却是难以估量的。

因为这个世间有些禁地需要特殊手段才能进入,比如说去地心、火焰洞等地炼制宝具,没有特别宝物根本不能入内。

甚至,一些传说中的绝地,神火燎天,孕育有绝世宝物。若是掌握有避火珠,可以深入进去,进行采摘。

而避火珠、定风珠这类奇珠,极其罕见,多少年都难以出现一枚,注定了它们会成为稀世之宝。

而且,它们其实还是炼制圣器的极佳材料,甚至炼制神器都要用到,为稀缺神料。

故此,那台上的话语刚一落毕,下方的人就立刻开始竞价,很热闹,无比激烈。

石昊眼巴巴的看着,直到这枚珠子被炒上天价,他才咋舌,并没有参与进去,因为他还用不上避火珠没有必要去跟着疯狂。

最终,一个女子清脆的声音响起,开出了一个惊人的天价以一堆原始符骨竞拍到这枚奇珍。

“是十六公主,人皇最喜欢的女儿。”有人惊呼,看到了最终的胜出者。

那是一个身穿黄色衣裙、明眸皓齿的少女非常漂亮,肤色白皙,此时手持赤红的珠子,流动绚丽的赤霞,两者交相辉映,有一种十分惊人的美感。

众人叹息谁能与皇族比?他们的底蕴最深厚,绝不差精璧,事实上拍卖到了这个份上,最后都以原始符骨当作货币了,这更直观。

“可惜了,又一件珍品出现,结果我等没有实力拿下啊。”

“别急,肯定还有好东西,这次的拍卖会规格极高精心准备了好长时间,集合了一大批宝物。”

人们私下议论,热情高涨都在期待更好的珍品出现。

当然,接下来出现的东西底价也将会越来越贵了,不一定好东西都留到最后,但相对来说还是有这种趋势的。

很快一把天魔伞呈现出来,通体乌光闪烁,阴气森森,这个东西很强大,是王侯兵器中的珍品,引发几位老王竞逐。

年轻一代也只能看着,没有参与,一是价格不菲,而是那几个老头子在争夺,小辈参与进去有点不太好,同在皇都,各族也是有些情分的。

很快,天罗伞被一位中年王爷竞拍走。

“小塔,你就不能给我点建议与参考吗,到底什么东西价值高,免得我错过啊。”石昊传音道。

小塔沉默,并没有给予回应。

熊孩子撇嘴,就不信它能沉得住气,有好东西出现,在它需要的时候肯定会出言的。

“下面我们展现给大家的是一张上古遗留下来的宝符,非常难得,请大家上眼。”台上一身红衣的妖艳女子笑道,向众人介绍。

这是一张土黄色的纸张,很具有柔韧性,无论怎么折都无损,且弹在上面铿锵作响,如金石般。

“有人应该认出了吧,这是缩地符,贴在脚下,可以缩地成寸,比之鸟雀飞行的还要快,乃是上古大能祭炼出来的东西,而今少有人能炼制了。”女子补充道。

许多人感兴趣,都露出关注的神色,缩地符等若是逃命符,若是被仇人追杀,可借助此符迅速逃走。

不过皇都中的一些贵族并不是多么感兴趣,显赫的身份决定了他们的选择,少有狼狈逃命的时候,他们喜欢进攻型法器。

事实上,若是论保命的东西,他们也不会少。

石昊动心,他曾在百断山截获过一枚缩地符,只是用到最后已然碎掉了,觉得很实用,想在此竞拍下来。

“三千精璧起拍。”

“四千精璧!”

……

一时间,大厅中竞拍声此起彼伏,虽然不像避火珠那么抢手,但也有一些人想拿下。

“将那我那块海蓝髓剁下拳头那么大一块,参与竞拍。”石昊吩咐,让拍卖行的人帮忙。

当看到那么大一块海蓝髓再次被搬出后,众人无言,这……简直就是个暴发户,也忒欺负人了。

事实上,这么大一块,只有小部分属于石昊了,余者都花费出去了,只不过还没有切割,留待最后进行。

海蓝髓也算是奇珍,拳头那么大一块价值连城,能提升很多件宝具的品阶,自然算是一种极度的挥霍。

结果,无人与他争,石昊用拳头大的一块宝髓,以暴发户的形式砸翻了一群人,如愿得到缩地符。

“唉,这样会不会让人眼红啊,到时候得想个办法。”石昊发愁。

“但请放心,最起码在皇都中没人敢乱打主意,我们骨蛊拍卖行还是有些能力的,保证贵宾的安全。”一位负责人将缩地符送来时,做出这样的承诺。

事实上,石昊拍的东西也不算扎眼,相对那些神物来说还算普通。

当然,前提是无人知道莹白骨块与原始真解有莫名联系。

“诸位是不是觉得平淡了,那么再让我们再次开启一番高潮,接下来要拍卖的东西我想很多人都会感兴趣。”红衣女子娇笑,吊足了人们的胃口。

她取出一块兽皮,很暗淡,并无什么出奇之处,然而当她抖手一甩披在自己的身上后,奇异的事情发生了,那具玲珑起伏的躯体消失。

“咦竟然不见了。”

“怎么回事,这是什么宝具?”

众人惊讶,都颇感兴趣。

刷的一声当红衣女子再现时,她已经将那块兽皮放下,娇笑道:“各位可猜出这是什么?”

“难道是……虚空兽皮?”有人颤抖。

众人闻言,全都吃惊,而后引发轰动,这可是稀珍材料实在太罕见了。

虚空兽对于当世人来说,应该不可见了,就是在上古年间也极其稀少,价值无量。

任何一块虚空兽皮出世,都可以卖到天价,因为它天然的特性就足以成为一件瑰宝,就更不用说后期加工了。

它能让人藏于虚无间,根本无从探查,祭炼得当更是穿梭虚空中,堪称瑰宝!

当初,石昊在百断山时曾见到火灵儿取出这样一张虚空兽皮属于火国人皇的法器,他们藉此躲避过了多场杀劫。

让一国人皇都收藏的器具,怎会是凡品?

\section{人皇}

当然,那一张很大远胜过眼前这张,这块皮只有一米见方,承载力有限。

“虚空兽皮我就不用多说了,它有什么神能,想必大家都清楚。”看台上的红衣女子爆出惊人的底价。

人们难以平静,这样一张兽皮的妙用太多了,可以躲避大劫,站在上面,就能冲入虚空,藏于虚无中。

而若是落在杀手的手里,那就更加可怕了,何处不能去,何人不敢刺杀,简直防不胜防。

当然,若是结合其他天材地宝祭炼,会更加宝贵,造就出稀世法器。

石昊眼红,开始报价,最后发现在皇都这个地方,他以海蓝髓砸人的暴发户表现不够看,比他财大气粗的人实在不少。

有人直接亮出几块稀有材料,光华闪闪,耀的人睁不开双眼。

还有人取出一大块地髓,是号称可以延长生命的东西。

石昊惊然发觉,不仅王侯在竞拍,还有一些陌生的年轻面孔,有纯血生灵的气息,他一阵头大,争不过了。

“看来,人皇大寿着实吸引来了不少来头很大的强者啊。”

最后,这块虚空兽皮落入一个神秘人的手中,他躲在贵宾室,并未暴露身份,给出的价格让所有人都无力。

十几种稀有材料,包括蓝灵土、紫晶等,全都很罕见,价值惊世。

“让你们先拼,等遇上好东西时,你们就无力了。”石昊这样安慰自己。

高潮过后,自然是要陷入一段平淡期,紧接着被展示的是一个拳头高的黑色小人,不知是什么金属,没有光泽,暗淡无光。

看不出什么特别之处,只是让人觉得,这是一件古物,而且岁月应该相当的久远。

“这是什么?”有人问到。

“这是一件上古法器,久远到我们很多人都可能难以追溯其源头。”红衣女子笑着介绍。

经历一系列拍卖,人们渐渐了解到了她的风格,一听这样的话语,有人立刻打趣道:“看来是一件古董,估计也仅止于此了,如果足够珍贵,你早就不吝赞美了。”

“估计是这样,也只剩下古老这两个字能夸赞了吧?”一些人笑道。

“话不能这么说,这是上古遗留下的法器,据猜测,这可能是一件至强的战偶,若是能修复,说不定会有大用。”红衣女子笑道。

\section{原本是}

“战偶,一旦破损,我看当世没有几人能修复吧,如果真是上古圣品,你们也不会这样拍卖了。”很多人摇头。

这个拳头高的小人暗淡无光,被送进贵宾室,给众人观看,亲手触摸。

很可惜,它虽然材质很坚硬,但是符文破碎了,没有多少保留下来,而且上面裂痕很多,像是随时会碎掉。

众人摇头,这个东西在上古时也许有些来头,但是遗留到现在早已失去了价值。

没有符文,也无神力波动,而且无论怎么催动都不能让它展现出威能,对于众人来讲,不说是废宝也差不多。

“唯一有些价值的也许只是它的材料吧,不过看起来也不是什么稀有的神料。”众人对它兴趣不大。

石昊持在手中观看了一会儿,入手很沉,冰冷而有质感,但他也感应不到什么特别之处。

就在这时,小塔突然开口,道:“竞拍下这件东西,足够你受用终生,我都有些舍不得吃掉,有点可惜啊。”

石昊心惊,小塔的来历何等的恐怖,被它这样评价的东西岂是凡物,绝对是人间至宝!

\chapter{中国科学院太阳活动重点实验室举行2013年学术年会}

12月6日至8日,中国科学院太阳活动重点实验室在北京举行2013年学术年会。参加会议的有实验室全体科研人员、研究生和部分退休老专家,以及来自南京大学、北京师范大学、中国科技大学、中国石油大学、昆明理工大学、紫金山天文台、云南天文台、南京天文光学技术研究所、上海天文台、新疆天文台等单位的研究人员计130余人。国家自然科学基金委员会数理学部副主任兼天文处处长董国轩研究员、国家天文台郑晓年副台长、郝晋新副台长、综合事务部陆烨主任、技术发展部包曙东主任、人力资源部杜红荣主任、财务资产部左成刚主任、科技计划处盘军处长、战略规划办公室赵冰主任、质量与条件保障处李久利处长、人事教育处田斌处长、资产处葛春波处长等有关职能部门负责人参加了会议。与会领导对实验室在过去一年所取得的各项成绩给予了充分肯定,同时,对实验室下一步的发展提出了具体要求,争取在明年5年一度的实验室评估和CSRH项目验收两个关键点上取得好成绩,并为下一步申请成为国家重点实验室奠定良好基础。 

本次会议主要围绕太阳物理新型探测技术和方法研究、太阳、类太阳恒星物理观测研究、太阳爆发非热过程研究、太阳等离子体物理研究、太阳磁场及太阳活动的多波段观测研究、太阳活动周几太阳活动预报研究等方向展开深入讨论。共有41位研究人员在大会上做了口头报告,8位研究人员提交了张贴报告,展示了基于国内设备并结合国内外空间观测资料获得的研究成果以及相关的理论研究进展。正在实验室访问的俄罗斯著名学者Chernov博士也参加了会议并做了学术报告。 

实验室主任颜毅华研究员作了实验室工作报告,对过去一年的各项工作进行了全面总结,并对下一年的工作计划及未来发展规划进行了详尽阐述。最后,实验室副主任张军研究员对本次会议做了总结。整个年会讨论激烈,气氛活跃,大家就太阳物理各研究方向的前沿科学问题进行了充分交流和讨论。